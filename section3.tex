\section{Project Objectives}
The key innovation in the approach that DEIS takes to address the dependability problem of CPS is the concept of Digital Dependability Identity (DDI), which have been outlined by key partners of DEIS in \cite{}. 
In general, a Digital Identity is defined as \emph{the data that uniquely describes a person or a thing and contains information about the subject's relationships} \cite{}. 
To this end, it contains all attributes that characterise the object, how the object can be accessed by whom, and how it may interact with other objects. A DDI contains all the information that uniquely describes the dependability characteristics of a system or component. This includes two key aspects: (a) attributes that describe the system’s or component’s dependability behaviour, such as faults and possible fault propagations through the CPS architecture, which can be described using concepts from the theory of safety contracts; and (b) requirements on how the component interacts with other entities in a dependable way, described in terms of the level of trust and assurance.

A DDI is therefore an evolution of current modular dependability assurance models of a system. It is produced during design, issued when the component is released, and then continuously maintained over the complete lifetime of a component or system. DDIs are used for the integration of components into systems during development as well as for the dynamic integration of systems into "systems of systems" in the field. The main difference in the two cases lies mainly in the degree of automation that integration of DDIs requires: for instance, while a manual process might be sufficient to ensure dependability requirements are met at the integration interface between silicon providers and first-tier suppliers, the dynamic integration of systems of systems in the field requires a fully automated evaluation of DDIs in order to ensure a dependable collaboration.

Clearly, a DDI is potentially a very useful digital artefact — it is a living dependability assurance case, the utility of which spans from component design to in-the-field operation of a CPS. However, the production and use of DDIs for heterogeneous systems poses a number of significant technological and engineering challenges that are pertinent and important in industry today and motivate the objectives of this project.

\subsection{Universal exchange of dependability 
information}


\subsection{Efficient dependability assurance across industries and value chains}


\subsection{Dependable integration of systems in the field}

