\section{Introduction}
It is expected that in the future, the physical and digital worlds will merge into a largely connected globe. This is backed by the emergence of notions such as Cyber-Physical-Systems (CPS). 
%At the moment, CPS technologies have been applied in a broad range fo domains, including smart grids, smart homes, smart health, intelligent transportation, etc. 
CPS harbour the potential for vast economic and societal impact in domains such as automotive, health care and home automation. At the same time, if these systems fail, they may cause harm and lead to temporary collapse of important infrastructures, with catastrophic consequences for industry and society. 
Therefore, in order to realise the full potential for innovation of CPS, it is important to ensure the dependability of CPS.
% - especially in relation to security and safety, as security and safety are important aspects in CPS, breaches to any of them can cause catastrophic events.

CPS are typically loosely connected and come together as temporary configurations of smaller systems which dissolve and give place to other configurations. Therefore, the configurations a CPS may assume over its lifetime are unknown and potentially infinite. Thus, currently available approaches are not possible to assure the dependability of CPS and it is a grand technology challenge to address the dependability of CPS. 

The DEIS project identifies this challenge and takes a first step towards dependability assurance of CPS by focusing on system safety and security, since assuring safety of such systems is an indispensable prerequisite in order to realise the economic and social potential of CPS.




