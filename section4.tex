\section{Project Objectives}
\label{section4}
Based on the identified challenges and the project concept, the objectives of the DEIS project are set out as the following.

%\subsection{Objective 1. An open dependability exchange (ODE) metamodel and a universal format for specifying DDIs}
\subsubsection{Objective 1. An open dependability exchange (ODE) metamodel and a universal format for specifying DDIs} 
Based on existing work, DEIS will produce an Open Dependability Exchange (ODE) metamodel. ODE provides the means to express, connect and communicate dependability information.  
With ODE, it would be possible to specify the level of trust of assured dependability properties with respect to the trust of the issuer and to the trust level of the promised services during field operation. DDIs should also be generated based on the information defined in ODE. 
%Where ODE may contain detailed intellectual property, DDIs will be defined at a more abstract level. 
DDIs will also be formalised in order to support their semi-automated evaluation.

Measurable sub-objectives of objective 1 are:
\begin{enumerate}
	\item Definition of the Open Dependability Exchange (ODE) metamodel
	\item Definition of general form of Digital Dependability Identity (DDI)
	\item Tooling support for the manual modelling of DDIs
	\item Tooling support to check the validity of DDIs
\end{enumerate}

\subsubsection{Objective 2. A framework for the creation and modular synthesis of DDIs}
Once an appropriate format for the ODE and DDIs is defined, DEIS will provide support for the creation and modular synthesis of DDIs from existing dependability information. Such support is a prerequisite for the practical applicability of the approach. Thus a framework that serves such purpose will be developed, covering the following sub-objectives:

\begin{enumerate}
	\item Tooling support for expressing existing dependability models in ODE-compliant format
	\item Algorithms and tooling support for synthesis of DDIs
	\item Algorithms and tooling support for integration of DDIs into the dependability assurance cases
	\item Algorithms and tooling support for change-impact analysis on DDIs
\end{enumerate}

\subsubsection{Objective 3. A framework for the in-the-field dependability assurance in CPS}
A framework which enables the dependable integration of open CPS is required. Such framework consists a centralised DDI registry which is publicly available on-the-cloud. By using the centralised DDI registry, system manufacturers can check if their systems can be dependably integrated with already existing systems. Beside the centralised DDI registry, the framework should also enable on-board evaluation. With on-board evaluation, systems carry DDIs with them and evaluate if they can collaborate with each other in the field.

The framework covers the following sub-objectives:

\begin{enumerate}
	\item Development of infrastructures for evaluation of integration of new systems in the field
	\item Development of algorithms for the on-board evaluation of DDIs
\end{enumerate}

\subsubsection{Objective 4. Development of autonomous and connected CPS use cases for different application domains, and validation of applicability and scalability of the DDIs}
The scope of the project and the technology it develops is wide reaching and fundamental for CPS and the industries involved in the project (road transport, railway, healthcare). As such, the project results are expected to create significant impact. For this reason, it is a further objective of the project to validate the results in four realistic scenarios based on representative projects. %The first two scenarios are directly related to new systems that can achieve radical cost reduction for automated driving systems, maintaining and improving dependability via DDIs. A further validation will take place in a railway scenario in context of the European Train Control System (ETCS) with the primary goal being to evaluate the applicability of the results across different application domains. The fourth use case is targeting the medical domain with management of patient-related information for clinical decision support.

The studies of the four scenarios covers the following sub-objectives:

\begin{enumerate}
	\item Evaluation of effectiveness of approach
	\item Evaluation of applicability across industries
	\item Evaluation of runtime mechanisms
	\item Evaluation of systems produced in four case studies
\end{enumerate}