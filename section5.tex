\section{Related Work}
In order to ensure successful project results, the project will not aim at developing an entirely new solution. In fact, the project will use, wherever appropriate, the results from other previous and current projects. In particular, projects that are related to DEIS are: 
\begin{itemize}
	\item VETESS: Verification and Testing to Support Functional Safety Standards
	\item SPES XT: Software Platform Embedded Systems
	\item SAFECER: Safety Certification of Software-Intensive Systems with Reusable Components
	\item CESAR/CRYSTAL: Cost Effective Small AiRcraft/CRitical sYStem engineering AcceLeration
	\item SAFE: Safe Automotive soFtware architEcture
	\item EMC$^2$: Embedded Multi-Core systems for Mixed Criticality applications in dynamic and changeable real-time environments
	\item SafeAdapt: Safe Adaptive Software for Fully Electric Vehicles
	\item OPENCOSS: Open Platform for EvolutioNary Certification Of Safety-critical Systems 
	\item D-MILS: Distributed MILS for dependable information and communication infrastructures
	\item MAENAD: Model-based Analysis \& Engineering of Novel Architectures for Dependable electric vehicles
	\item ATESST2: Advancing Traffic Efficiency and Safety through Software Technology phase 2
	\item COMPASS: Comprehensive Modelling for Advanced Systems of Systems
\end{itemize}

The research in DEIS can also be based upon different existing approaches, like Component Fault Trees \cite{Kaiser2003} and HiP- HOPS \cite{Papadopoulos1999} for dependability analysis, GSN \cite{kelly2004goal} for specifying safety cases, SACM \cite{sacm2} for specifying structured assurance cases,  or ConSerts \cite{Schneider2003} as a starting point for runtime certification. All of these approaches were defined by partners involved in DEIS and have proven their value in many practical applications. The fundamental competence and previous work results provided by the partners involved in DEIS therefore build a sound basis, which gives confidence that the project objectives are achievable within the proposed time and budget of the project.